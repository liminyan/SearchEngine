%!TEX program = xelatex
\documentclass{article}
\usepackage{ctex}
\usepackage{url}
\usepackage{geometry}
\usepackage{graphicx}
\usepackage{abstract}
\usepackage{amsmath,amssymb,bm}
\usepackage{cite}
\usepackage{clrscode}
\usepackage{enumerate}
\usepackage{authblk}
\usepackage{listings} 
\usepackage{graphicx}
\usepackage{pythonhighlight}
\usepackage{float}
\usepackage[marginal]{footmisc}

\graphicspath{{figure/}}
\geometry{left=2.5cm,right=2.5cm,top=2.0cm,bottom=2.0cm}
\setlength{\parindent}{2em}
\pagestyle{plain}

\begin{document}

\title{搜索引擎PA\_2 \\[1ex]\begin{Large}——实验报告\end{Large}}
\author{闫力敏\thanks{清华大学计算机系.~ 学号:2015011391.~ 邮编:100084}}
\date{}

	\maketitle

	\section{概述}
		本次实验主要实现一个简单的搜索引擎,然后通过给定的Query集进行测试,最后将所得结果采用给定的DCG,N-Err以及Q-Measure指标评测
	\section{实验过程}
		利用lucene提供的框架,实现一个简单的搜索引擎,其中通过继承SimilarityBase类重载score 实现自定义排序BM25
		\par
		查询区域选取为title和content两个区域	,分别计算出相关的分数之后,按照3:1加权计算得到最终的结果
		$$BM25_{final} = BM25_{title} * 0.75 + BM25_{content} $$
		\par
		每次查询返回top20的规定结果(包括分数),提取出y\_pred,随后利用给定的ntcir14\_test\_label.txt数据文件提取出对应的y\_true,根据要求的评测指标进行计算。
	\section{实验结果}
		\begin{tabular}{|l|c|r|l|} 
		\hline 
		k &n\_dcg&q\_measure&n\_err\\
		\hline 
		20 &0.441 &0.329& 0.689\\
		10 &0.514 &0.412& 0.686\\
		5 &0.692 &0.592& 0.684\\
		\hline 
		\end{tabular}
		\par 
		top\_N的结果整体来说随着N的减少,DCG,Q-Measure,正相关,且比较明显,而N-Err负相关,但差异变化不大,总体来说还是比较符合预期结果的

	\section{总结}
		本次实验,总体来说还是比较简单的,主要的收获还是学到了一些比较有用的工具
	\bibliographystyle{unsrt}
	\bibliography{reference}
\end{document}

